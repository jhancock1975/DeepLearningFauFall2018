%%%%%%%%%%%%%%%%%%%%%%%%%%%%%%%%%%%%%%%%%
% Beamer Presentation
% LaTeX Template
% Version 1.0 (10/11/12)
%
% This template has been downloaded from:
% http://www.LaTeXTemplates.com
%
% License:
% CC BY-NC-SA 3.0 (http://creativecommons.org/licenses/by-nc-sa/3.0/)
%
%%%%%%%%%%%%%%%%%%%%%%%%%%%%%%%%%%%%%%%%%

%----------------------------------------------------------------------------------------
%	PACKAGES AND THEMES
%----------------------------------------------------------------------------------------

\documentclass{beamer}

\mode<presentation> {

% The Beamer class comes with a number of default slide themes
% which change the colors and layouts of slides. Below this is a list
% of all the themes, uncomment each in turn to see what they look like.

%\usetheme{default}
%\usetheme{AnnArbor}
%\usetheme{Antibes}
%\usetheme{Bergen}
%\usetheme{Berkeley}
%\usetheme{Berlin}
%\usetheme{Boadilla}
%\usetheme{CambridgeUS}
%\usetheme{Copenhagen}
%\usetheme{Darmstadt}
%\usetheme{Dresden}
%\usetheme{Frankfurt}
%\usetheme{Goettingen}
%\usetheme{Hannover}
%\usetheme{Ilmenau}
%\usetheme{JuanLesPins}
%\usetheme{Luebeck}
%\usetheme{Madrid}
%\usetheme{Malmoe}
%\usetheme{Marburg}
%\usetheme{Montpellier}
%\usetheme{PaloAlto}
%\usetheme{Pittsburgh}
%\usetheme{Rochester}
%\usetheme{Singapore}
%\usetheme{Szeged}
%\usetheme{Warsaw}

% As well as themes, the Beamer class has a number of color themes
% for any slide theme. Uncomment each of these in turn to see how it
% changes the colors of your current slide theme.

%\usecolortheme{albatross}
%\usecolortheme{beaver}
%\usecolortheme{beetle}
%\usecolortheme{crane}
%\usecolortheme{dolphin}
%\usecolortheme{dove}
%\usecolortheme{fly}
%\usecolortheme{lily}
%\usecolortheme{orchid}
%\usecolortheme{rose}
%\usecolortheme{seagull}
%\usecolortheme{seahorse}
\usecolortheme{whale}
%\usecolortheme{wolverine}

%\setbeamertemplate{footline} % To remove the footer line in all slides uncomment this line
%\setbeamertemplate{footline}[page number] % To replace the footer line in all slides with a simple slide count uncomment this line

%\setbeamertemplate{navigation symbols}{} % To remove the navigation symbols from the bottom of all slides uncomment this line
}

\usepackage{graphicx} % Allows including images
\usepackage{booktabs} % Allows the use of \toprule, \midrule and \bottomrule in tables
\usepackage{url}

%----------------------------------------------------------------------------------------
%	TITLE PAGE
%----------------------------------------------------------------------------------------

\title[Unsupervised Representation Learning with Deep Convolutional
Generative Adversarial Networks]{Presentation on ``Unsupervised Representation 
Learning with Deep Convolutional Generative Adversarial Networks,'' by Alec Radford, Luke Metz, and Soumith Chintala} % The short title appears at the bottom of every slide, the full title is only on the title page

\author{John Hancock} % Your name
\institute[FAU] % Your institution as it will appear on the bottom of every slide, may be shorthand to save space
{
Florida Atlantic University \\ % Your institution for the title page
\medskip
\textit{jhancoc4@fau.edu} % Your email address
}
\date{\today} % Date, can be changed to a custom date

\begin{document}

\begin{frame}
\titlepage % Print the title page as the first slide
\end{frame}

\begin{frame}
\frametitle{Overview} % Table of contents slide, comment this block out to remove it
\tableofcontents % Throughout your presentation, if you choose to use \section{} and \subsection{} commands, these will automatically be printed on this slide as an overview of your presentation
\end{frame}

%----------------------------------------------------------------------------------------
%	PRESENTATION SLIDES
%----------------------------------------------------------------------------------------

%------------------------------------------------
\section{Definitions} % Sections can be created in order to organize your presentation into discrete blocks, all sections and subsections are automatically printed in the table of contents as an overview of the talk
%------------------------------------------------

\begin{frame}
\frametitle{Background}
A generative adversarial network (GAN) is a neural network with two components:
\begin{itemize}
  \item a \textit{generator} that learns to transform vectors of random numbers into
    output values.  In the context of this paper, the outputs are images.  However, 
   researchers use GAN's where the generators create 
\end{itemize}
A deep convolutional generative adversarial network is
\end{frame}

%------------------------------------------------
\section{Subject}
\begin{frame}
\frametitle{Subject}

The authors of the paper make several contributions they...
\begin{itemize}
  \item invent an architecture for deep convolutional generative adversarial network (DCGAN),
  \item use the convolutional layer filters of trained DCGAN's discriminators as 
    feature extractors for doing classifications,
  \item demonstrate that after training the DCGAN, its filters learn how to
    represent images, and
  \item present a method of doing vector arithmetic using DCGAN inputs to do 
    inferences \emph{{\`a} la} Word2Vec \cite{word2Vec}.
\end{itemize}
\end{frame}

%------------------------------------------------
\section{Architecture}
\begin{frame}
\frametitle{Architecture}
A high-level overview of the architecture:
\begin{itemize}
  \item they use convolutions
  \item they use de-convolutions
  \item subsequent work such as the texts for this course 
   \cite{deepLearnR} \cite{deepLearnBookGenCh} suggests one
   should employ dropout, but we do not find that the authors of this paper use it
   when inspecting the code in \cite{dcganCode}.
\end{itemize}
\end{frame}

%------------------------------------------------

\begin{frame}
\frametitle{Architecture}
subsection{Deconvolution visualization}
The authors of the paper prefer the term, ``fractionally-strided.'' 

\end{frame}

%------------------------------------------------

\section{How did they do it?}
\begin{frame}
\frametitle{How did they do it?}
The code for this paper, as well as many others in the references and
that one may find in the course of research, is on Github in the dcgan\_code
project \cite{dcganCode}.

This code is a bit outdated, however the dcgan\_code Github project has a link
to the DCGAN-tensorflow \cite{dcganTf} project that we find more accessible.
\end{frame}

%------------------------------------------------

\begin{frame}
\frametitle{Datasets}
\begin{itemize}
  \item The authors use three datasets for training:
  \begin{itemize}
    \item Large Scale Scene Understanding (LSUN),
    \item Imagenet 1-K, and
    \item Faces.
  \end{itemize}  
  \item The authors two datasets for evaluating unsupervised learning:
  \begin{itemize}
    \item Canadian Institute for Advanced Research (CIFAR) 10
    \item StreetView House Numbers (SVHN)
  \end{itemize}
  \item Note: the authors mention that they heuristically removed duplicate
    images from LSUN to prevent the DCGAN from memorizing images.
\end{itemize}
\end{frame}

%------------------------------------------------

\begin{frame}
\frametitle{LSUN}
\begin{itemize}
 \item The authors used images of bedrooms from the LSUN dataset \cite{lsunDataset} as input to their model.
\end{itemize}
\end{frame}

%------------------------------------------------


\subsection{Run their MNIST Example}
\begin{frame}
\frametitle{How did they do it?}
The authors of the paper make several contributions:
Here reference github code implementation
How to run their mnist example:
\begin{itemize}
  \item Use AWS Ubuntu Deep Learning Instance
  \item \begin{itemize}
    \item Expensive $\approx$ \$0.65 per hour!
    \item Configure an alarm to shut the instance down after 3 hours!
    \end{itemize}
 \item Create virtual environment
 \item \begin{itemize}
    \item Use pip to install libraries
    \item force install of Theano 0.9.0 (pip install -I Theano 0.9.0)
    \end{itemize} 
 \item \begin{itemize}
    \item Paper dcgan\_code repository does not have MNIST data,
    \item Download MNIST data from \url{https://github.com/Manuel4131/GoMNIST/tree/master/data}, and change location in lib/config.py
    \end{itemize}
 \end{itemize}
\end{frame}

%------------------------------------------------
% https://ieee-dataport.org/sites/default/files/analysis/27/IEEE%20Citation%20Guidelines.pdf

\begin{frame}[allowframebreaks]
\frametitle{References}
\footnotesize{
\begin{thebibliography}{99} % Beamer does not support BibTeX so references must be inserted manually as below

\bibitem{repLearnDcgan} S. Chintala, L. Metz, and A. Radford, 
Unsupervised representation learning with deep convolutional generative 
adversarial networks.  2016. [Online]. Available: arXiv:1511.06434v2 [cs.LG]

\bibitem{dcganCode} S. Chintala, L. Metz, and A. Radford, dcgan\_code. (2016,
May).  Available: \url{https://github.com/Newmu/dcgan\_code}. [Accessed Nov. 19,
2018].

\bibitem{lsunDataset} Large Scale Scene Understanding Challenge (2017, July).
Available: \url{http://lsun.cs.princeton.edu/2017/}. [Accessed Nov. 19, 2018].

\bibitem{dcganTf} T Kim. (2018, August).  Available: \url{https://github.com/carpedm20/DCGAN-tensorflow}. [Accessed Nov. 19, 2018].


\bibitem{gan} Y. Bengio, A. Courville,I. Goodfellow, M. Mirza, S. Ozair,  
J. Pouget-Abadie,  D. Warde-Farley, B. Xu, Generative adversarial nets. 2014. 
[Online]. Available: arXiv:1406.2661v1 [stat.ML]

\bibitem{word2Vec} T. Mikolov, I. Sutskever, K. Chen, G. Corrado, J. Dean,
``Distributed Representations of Words and Phrases and their Compositionality,''
Advances in Neural Information Processing Systems 26 (NIPS 2013), 2013. 
[Online] Available: \url{http://papers.nips.cc/paper/5021-distributed-}representations-of-words-and-phrases-and-their-compositionality.pdf

\bibitem{slidetemplate} Creodocs Limited,``Beamer Presentation,''
\emph{latextemplates.com}, 2018. [Online], Available: 
\url{http://www.latextemplates.com/templates/presentations/1/presentation\_1.zip}. [Accessed Nov. 10, 2018].

\bibitem{ieeeStyle} IEEE, Piscataway, NJ, USA. \emph{IEEE Editorial Style Manual}. 2016.
[Online]. Available: \url{http://ieeeauthorcenter.ieee.org/wp-content/uploads/IEEE\_Style_Manual.pdf}, Accessed on: Nov. 11, 2018.

\bibitem{unsupVideo} Y. LeCun, ``Unsupervised Representation Learning,''
2017. Accessed on: Nov 11, 2018. [online]. Available: \url{https://www.youtube.com/watch?v=ceD736_Fknc}  

\bibitem{deepLearnR} F. Chollet, and J.J. Allaire.  
\textit{Deep Learning with R}. Manning Publications 2018. [E-Book] Available: Safari
E-Book.

\bibitem{deepLearnBookGenCh} Y. Bengio, A. Courville, I. Goodfellow. (2016). ``Chapter 20 Deep Generative Models,'' 2016. [Online] Available: \url{https://www.deeplearningbook.org/contents/generative_models.html}. [Accessed: Nov. 20, 2018].
 
\end{thebibliography}
}
\end{frame}

%------------------------------------------------

\begin{frame}
\Huge{\centerline{The End}}
\end{frame}

%----------------------------------------------------------------------------------------

\end{document} 
