\documentclass{beamer}
\usepackage[latin1]{inputenc}
\usepackage{times}
\usepackage{tikz}

\usepackage{verbatim}
\usetikzlibrary{arrows,shapes}


\begin{document}
\begin{comment}
:Title: The TeX work flow
:Slug: tex-workflow
:Tags: Beamer, Diagrams
:Use page: 6

I spotted this well-known diagram of the TeX work flow in [1]_, where it was used to 
illustrate when a step-by-step presentation technique is appropriate. With Beamer 
and TikZ it is quite easy to gradually draw a diagram, since the ``\path`` construct
is overlay aware.  Download the PDF to see it in action.

.. [1] Veytsmann, B. (2006). `Design of Presentations: Notes on
       Principles and LaTeX Implementation`__. *The PracTeX Journal*, 4

.. __: http://www.tug.org/pracjourn/2006-4/veytsman-design/
\end{comment}

\begin{frame}
\frametitle{The \TeX\ work flow}


\tikzstyle{format} = [draw, thin, fill=blue!50!cyan!80]
\tikzstyle{medium} = [draw, thin, fill=red!40]
\tikzstyle{cost} = [draw, thin, fill=green!20]
\tikzstyle{input} = [draw, thin, fill=brown!10]
\tikzstyle{operator} = [circle, draw, thin, fill=green!20]

\begin{footnotesize}
\begin{figure}
\begin{tikzpicture}[thick]

\node[input] (noise) at (0,0) {Noise Vector};

\node[input] (dataset) at (0,2) {Dataset};

\node[format](generator) at (3,0) {Generator G};

\node[medium] (gOptimizer) at (6.25, 0) {Optimizer};

\node[format](discriminator) at (3,2) {Discriminator D};

\node[medium] (dOptimizer) at (3,4) {Optimizer};

\node[cost] (dCostDataset) at (6.25,3){D Cost Dataset};

\node[cost] (dCostGenerated) at (6.25,2){D Cost Generated};

\node[cost] (gCostGenerated) at (6.25,1){G Cost Generated};

\node[operator] (adder) at (5.5,4){+};

\draw[->] (noise) -- (generator);

\draw[->] (generator) -- (discriminator) ;

\draw[->] (gOptimizer) -- (generator);

\draw[->] (gCostGenerated) -- (gOptimizer);

\draw[->] (dataset) -- (discriminator);

\draw[->] (discriminator) -- (4.5, 2) |- (dCostDataset);
\draw[->] (discriminator) -- (4.5,2) |- (gCostGenerated);
\draw[->] (discriminator) -- (dCostGenerated);

\draw[->] (dCostGenerated) -- (8,2) |- (adder);
\draw[->] (dCostDataset) -- (8,3) |- (adder);

\draw[->] (adder) -- (dOptimizer);

\draw[->] (dOptimizer) -- (discriminator);

\end{tikzpicture}
\end{figure}
\end{footnotesize}
\end{frame}
\end{document}
