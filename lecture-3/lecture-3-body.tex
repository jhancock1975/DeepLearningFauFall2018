\section{Introduction}


Notes taken while working on lecture 3

\section{Main Body}
\subsection{Google Colaboratory}
This posting from a classmate in course discussion:

Christian Garbin
Sunday Aug 26 at 4:32pm
Google Colab (Links to an external site.)Links to an external site., an online Jupyter platform, also offers a free GPU (a Tesla K80 (Links to an external site.)Links to an external site.). It's limited to 12 hours of continuous runtime, but that's more than enough for experiments.

The catch: their Jupyter platform supports only the Python kernel.

However, one of the selling points of Keras is to offer the same (or very similar) API across different languages, so translating R Keras code to Python is a relatively simple task.

I translated the chapter 2 MINST example to Python here (Links to an external site.)Links to an external site., to test the performance with and without a GPU. With GPU enabled the code runs in half of the time.

The link above is a live notebook, in case you want to play with it.

On a side note, the Google Colab platform is a great playground. This is a short tutorial (Links to an external site.)Links to an external site. to get started, including how to use the GPU.

\section{Conclusions}

Lecture 3 the main thing we learned about is perceptron.

% Start of "Sample References" section

\section{References}

\begin{acks}
The author would like to thank the staff of the Computer Science Department
of Florida Atlantic University for their priceless guidance over the years.
\end{acks}

% Bibliography
\bibliographystyle{ACM-Reference-Format}
\bibliography{sample-bibliography}
